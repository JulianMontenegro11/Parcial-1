\documentclass{article}
\usepackage[utf8]{inputenc}
\usepackage[spanish]{babel}
\usepackage{listings}
\usepackage{graphicx}
\graphicspath{ {images/} }
\usepackage{cite}

\begin{document}

\begin{titlepage}
    \begin{center}
        \vspace*{1cm}
            
        \Huge
        \textbf{Parcial 1 - Calistenia}
            
        \vspace{0.5cm}
        \LARGE
        Informatica ll
            
        \vspace{1.5cm}
            
        \textbf{Edwin Julian Montenegro Pinzon}
        
        \textbf{cc. 1070306733}
            
        \vfill
            
        \vspace{0.8cm}
            
        \Large
        Despartamento de Ingeniería Electrónica y Telecomunicaciones\\
        Universidad de Antioquia\\
        Medellín\\
        Marzo de 2021
            
    \end{center}
\end{titlepage}

\tableofcontents
\newpage
\section{Introducción}\label{intro}
En este ejercicio se dará evidencia de lo que es un estudio completo a la hora de resolver un problema o al momento de ejecutar una acción, al realizar un previo análisis se puede ejecutar un plan de acción detallado para poder llegar al correcto resultado.

\section{Instrucciones} \label{contenido}
A continuación se entregarán una serie de instrucciones con el objetivo de llegar de una posición 1 inicial a una posición 2. La posición inicial consta de dos tarjetas con las mismas dimensiones posicionadas una sobre la otra, sobre ellas se encontrará una hoja en blanco.
\begin{itemize}
    \item tomar la hoja de papel.
    \item Poner la hoja al lado izquierdo de las tarjetas.
    \item Tomar las dos tarjetas con la mano derecha.
    \item Deslizar una tarjeta sobre la otra con el pulgar para separarlas.
    \item Poneritem dedo anular en medio de las dos tarjetas.
    \item Poner la parte inferior de la tarjeta externa sobre la hoja de papel.
    \item Unir las partes superiores de las tarjetas.
    \item Poner la parte inferior de la tarjeta interna en la hoja de papel.
    \item Asegurar que las dos tarjetas se sostengan desde la parte superior.
\end{itemize}


\section{Conclusion}
Podemos concluir que al tener un objetivo claro y al realizar un detallado análisis cualquiera puede llegar al resultado buscado siguiendo un conjunto de instrucciones.



\bibliographystyle{IEEEtran}


\end{document}
